\documentclass[letterpaper]{article}
\usepackage{graphicx}
\usepackage[export]{adjustbox}
%\usepackage{color}
\usepackage{amsmath, amssymb}
\usepackage{tikz}
\usetikzlibrary{patterns, shadows, positioning}
\usetikzlibrary{shapes.geometric}
%\usepackage{pifont}
\usepackage{fourier-orns}
\usepackage[hmargin=0mm,vmargin=0mm, letterpaper]{geometry}
\usepackage{tabularx}
\usepackage{epsdice}
\usepackage{url}

%\setlength{\parindent}{0cm}

\definecolor{titlebg}{rgb}{30, 30 , 30}

% https://latexcolor.com/
\definecolor{capri}{rgb}{0.0, 0.75, 1.0}
\definecolor{royalazure}{rgb}{0.0, 0.22, 0.66}
\definecolor{lightskyblue}{rgb}{0.53, 0.81, 0.98}
\definecolor{qr}{rgb}{1,1,1}


\def\cardwidth{2.5in}
\def\cardheight{3.5in}

\def\ruleswidth{2in}

\def\shapeCard{(0,0) rectangle (\cardwidth, \cardheight)}

\tikzset{%
  square/.style={regular polygon,regular polygon sides=4}
}


\newcommand{\rulescard}[1]{
\begin{tikzpicture}

  \draw[white, line width=0pt] \shapeCard;

  \node[below right] at (0.1*\cardwidth,\cardheight-0.1*\cardwidth) [text width = 0.8*\cardwidth] {\setlength{\parindent}{15pt}#1};
   
\end{tikzpicture}}


\begin{document}

% https://tex.stackexchange.com/questions/56101/spaces-between-rows-and-cols-in-a-table-better-ways-than-mine
\setlength\tabcolsep{0pt}
\renewcommand{\arraystretch}{0}






% https://socratic.org/questions/how-do-you-write-all-permutations-of-the-letters-a-b-c-d#419492




\def\titlecard{%
  \textbf{Talavera} is a game of fulfilling customer orders in 4
  rounds of two phases: \textbf{draft} cards and \textbf{place} tiles. When
  all the cards have been drafted and placed, players \textbf{score}
  according to the number of fulfilled orders and wasted tiles.
  
  One side of each card has four numbered rectangles and represents a
  unique customer \textbf{order}. The other side has four square
  \textbf{tiles} grouped by color which players will draft to
  fulfill that order.

  \begin{centering}
    \includegraphics[width=0.5in]{qr.png}
    
    \footnotesize{\url{https://jlericson.com/talavera/}}
  \end{centering}
}

\def\draftcard{%
  \textbf{Draft Phase}
      
  Shuffle the cards and give each player an order card. Whichever
  player has the smallest number (reading clockwise starting with
  yellow) on their order is the first player.

  Place four cards on the table with the order side up to form the
  market. The first player flips two cards and selects one to
  keep. The second player flips the other two cards and selects one of
  the three cards remaining in the market.
      
  Take turns drafting one card until the market is empty and you have
  two cards to place.  }

\def\placementcard{%
  \textbf{Place Phase}
      
  Tuck each drafted card under one corner of your order card so that
  all the tiles of that color are visible.
  
  \textbf{Note}: Only one tile color on each drafted card counts
  toward your score.

  You may place both cards under the same color or different
  colors. You may also add tiles to a color you have already
  placed. Do not place a card under a color that isn't represented by
  at least one tile on that card.

  Swap who drafts fist and continue drafting tiles until
  all the cards have been placed.
}

\def\scoringcard{%
  \textbf{Score}
      
  In order to fill an order, there must be \textit{at least} as many tiles of
  that color as indicated by the number on the order.

  \begin{itemize}
    \setlength\itemsep{0em}
  \item Each fulfilled order is 3 points.
  \item Subtract 1 point for each excess tile. Orders can't go below 0.
  \item If there are fewer of a color than required, the order is worth 0 points.
  \end{itemize}

  In case of a tie, play best two out of three games.
}

\def\logocard{%
  \begin{centering}
    \includegraphics[width = 3in, angle=90]{talavera.jpg}
  \end{centering}
}

\begin{table}[ht]
  \centering
  \begin{tabular}{|cc|c}
    \hline\\
    \rulescard{\scoringcard} &
    \rulescard{\titlecard} &
    \rulescard{\logocard} \\
    \hline
    \rulescard{\scoringcard} &
    \rulescard{\titlecard} &
    \rulescard{\logocard} \\
    \hline
    \rulescard{\scoringcard} &
    \rulescard{\titlecard} &
    \rulescard{\logocard}\\
    \hline
  \end{tabular}%
  \hskip\headheight
\end{table}

\begin{table}[ht]
  \centering
  \begin{tabular}{ccc}
%    \hline
    \rulescard{} &
    \rulescard{\draftcard} &
    \rulescard{\placementcard} \\
%    \hline
    \rulescard{} &
    \rulescard{\draftcard} &
    \rulescard{\placementcard} \\
   \rulescard{} &
    \rulescard{\draftcard} &
    \rulescard{\placementcard}
  \end{tabular}%
  \hskip\headheight
\end{table}




\end{document}
